%! Compiler = xelatex
%! Author = Andrew Selvia
%! Date = 2020.10.8

% Preamble
\documentclass[11pt]{article}

% Use SF Pro Text per https://latex.org/forum/viewtopic.php?t=33196 & https://gist.github.com/Pni0/2923266
\usepackage{fontspec}
\setmainfont{SF Pro Text}
\usepackage{sfmath}
\everymath={\textsf{}}

\usepackage{hyperref}
\hypersetup{
colorlinks=true,
filecolor=magenta,
linkcolor=blue,
urlcolor=cyan
}
\usepackage{enumitem}

\begin{document}
    \begin{itemize}
        \item Title: \(E_{in-style}\)
        \item Team:
        \begin{itemize}
            \item Charlie Brayton (014559415)
            \item Mohit Patel (014501461)
            \item Andrew Selvia (014547273)
            \item Dylan Zhang (013073437)
        \end{itemize}
        \item Data: \href{https://github.com/zalandoresearch/fashion-mnist\#get-the-data}{fashion-mnist}
        \item Idea:
        Put simply, we intend to classify and augment pictures of clothing items.
        Specifically, we will leverage the fashion-mnist dataset to explore various machine learning topics: multi-class classification, supervised learning, and generative adversarial networks (GANs).
        Luckily, we will be able to build upon prior work by many researchers to catalyze our own research.

        For instance:
        \begin{itemize}
            \item Students like us across the world have used fashion-mnist to gain experience with classification and clustering (such as \href{https://cnedwards.com/files/Edwards_Yen_Final_Project.pdf}{this student paper from the University of Tennessee, Knoxville}).
            \item At the same time, professional researchers at Google Brain have used fashion-mnist to explore various GANs in \href{https://arxiv.org/abs/1711.10337}{Are GANs Created Equal? A Large-Scale Study}.
            \item Finally, fashion-mnist has been used to demonstrate \href{https://arxiv.org/pdf/1710.10766.pdf}{resiliency against adversarial input} which has important ramifications for the trustworthiness of ML solutions as they slowly seep into every crevice of modern life.
        \end{itemize}

        Initially, our focus will be on acclimating ourselves to the dataset and techniques employed in existing classification papers.
        We each intend to independently survey one classification algorithm and share with the group.
        Assuming we find success, we then plan to split up based on topical interest.
        Dylan and Mohit have expressed interest in using ML to add color to the grayscale images in the fashion-mnist dataset.
        Andrew and Charlie have expressed interest in exploring GANs and adversarial input defense strategies.
        We may attempt to utilize the SJSU HPC system, especially if it offers GPUs capable of running our experiments.
    \end{itemize}
\end{document}